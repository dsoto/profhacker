\documentclass{article}
\usepackage{color}
\usepackage[nohead, margin=1.5in]{geometry}
\definecolor{answer-color}{rgb}{0.0, 0.5, 0.0}

% creates centered heading for sections
\newcommand{\chead}[1]
{\begin{center}\large\textbf{#1}\end{center}}

% solution
\newcommand{\solution}[1]
{\section*{}
\vspace{10pt}
{\color{answer-color} #1}
}

\newcommand{\ufrac}[2]{\frac{\textrm{#1}}{\textrm{#2}}}


\begin{document}

\chead{ENSP 330 Homework 4}
\chead{Due Date: 21 Nov 2013}
\hrule
\vspace{10pt}


The problem set you turn in should be both legible and the methods and
assumptions you are using must be clear.  I strongly suggest that you
use diagrams where appropriate and clearly state your assumptions.
Typesetting of problem sets can increase the readability.  The homework
will be graded on how well you communicate your method as much as the
correctness of the result.

Due Date: Thursday, November 21st

<% set QAs = d['330-HW-questions.tex|idio|t'] %>

% iterate over assigned numbers and pull in sections with labels q[i]
% and a[i]

<% for i in [1, 3, 5] %>

\section{}
<< QAs['q%s' % i] >>

\solution{<< QAs['a%s' % i] >>}

<% endfor %>


\end{document}
