%%% 'q1'

Sea Level Rise \\

The IPCC report states that the mean rate of averaged global sea level
rise was 3.2 mm per year between 1993 and 2010.

\paragraph{a.} How much did the sea level rise in total between 1993 and 2010? (5
pts)

\paragraph{b.} If this rate continues, how much higher do we expect the sea level to
be in the year 2050? (5 pts)

%%% 'a1'
To get a total amount, we multiply the rate by the time.
$$ \ufrac{3.2 mm}{year} \cdot (2010 - 1993) years = 54.4 mm $$
This is 5 cm or about two inches.

We calculate the rise from this year 2013 as we did above
$$ \ufrac{3.2 mm}{year} \cdot (2050 - 2013) years =  118 mm $$

%%% 'q2'

Mass of carbon in the atmosphere \\

The effective volume of the atmosphere is about 4.2 billion cubic
kilometers.  If the density of carbon dioxide is 1.9 kg/cubic meter and
the amount of carbon in the atmosphere is 397 ppm, what is the total
mass of carbon?  (Note that the actual volume of the atmosphere is different
than the effective volume.  The effective volume allows us to make
calculations more easily.)

\paragraph{a.} Convert 4.2 billion cubic kilometers to cubic meters. (5 pts)

\paragraph{b.} In your own words, what does 397 ppm mean? (5 pts)

\paragraph{c.} Calculate the mass of carbon in the atmosphere. (5 pts)

%%% 'a2'
To convert, we use 1000 meters equals 1 km and remember that this is a
cubic measure
$$ 4.2 \cdot 10^9 km^3 \cdot \frac{(1000m)^3}{(1km)^3} = 4.2 \cdot
10^{18} m^3$$

For every unit volume of atmosphere, if we divide it into 1 million
equal volumes, with only one gas in each, 397 of those tiny volumes will
be carbon dioxide.
$$ \ufrac{397 parts carbon dioxide}{1 million parts atmosphere} $$

We multiply the total mass of the atmosphere by the fraction of carbon
dioxide by the density of carbon dioxide.
$$ 4.2 \cdot 10^{18} cubic meters
\ufrac{397 parts carbon dioxide}{1 million parts atmosphere}
\ufrac{1.9 kg}{cubic meter} = 3168 \cdot 10^{12} kg$$


%%% 'q3'

Many of us will buy a Christmas tree this year.  What impact on climate
change does your purchase of a Christmas tree make? (10 pts)

%%% 'a3'
This question asks you to think about the plausible impacts of growing
and then consuming a tree.  There are several impacts to consider.
During the tree's lifetime, it will sequester carbon in the wood.  When
the tree is discarded, it will begin to decompose and release that
carbon dioxide.  You may also consider the energy cost of felling and
transporting the tree.


%%% 'q4'

Name three personal decisions you could make to reduce your personal
carbon dioxide emissions. (10 pts)


%%% 'a4'

There are many possible answers to this question.  Your options include
using more efficient technologies for transportation or energy, changing
your purchasing patterns to buy products with lower carbon footprints,
and making lifestyle changes that lower your use of carbon such as diet
or your mode of transportation.


%%% 'q5'

What could you do to help reduce the carbon dioxide emissions of the United
States as a whole? (10 pts)

%%% 'a5'
In order to affect the carbon emissions outside of your direct control,
you have to exert either political or economic pressure.  You can
communicate with lawmakers to announce your support for the policies and
educate others about laws or policies.  Economically, if there is a
company with practices you support, you can use that companies services
or products.

%%% 'q6'

Estimate to the nearest hour the time you spent on this homework. (5
pts)

%%% 'a6'

Your answer will vary.  I was hoping for 3--6 hours.
